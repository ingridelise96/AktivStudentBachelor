\cleardoublepage

\pagenumbering{roman} \setcounter{page}{1}
\chapter*{Sammendrag}

\meta{
Sammedraget er hele rapporten komprimert til max 1 side. Sammendraget skal gi leseren et godt og tilnærmet komplett bilde av innholdet i dokumentet. Akademiske sammendrag kalles på engelsk for ``Abstract'', og i mer kommersielle sammenhenger heter det gjerne ``Executive Summary''. I det siste tilfelle har sammendraget som hensikt å gi ledelsen i en bedrift nok informasjon til å ta økonomiske og/eller administrative avgjørelser\dots uten å lese hele rapporten (!). Tradisjonelt  blir sammendraget formattert som et sammenhengende avsnitt. I et bachelorprosjekt,
 vil hovedformålet være å gi leseren (kanskje i første rekke sensor?) et informativt (og appetittvekkende) bilde av prosjektet. Det er ikke vanlig å bruke litteratur- eller kryssreferanser i sammendraget. Som en regel kan vi si at alt som står i sammendraget, kan det leses mer om i rapporten. Dermed blir utfordringen å belyse alle viktige hovedpunkter, kort og presist. For denne rapporten, kan det f.eks. bli som dette:
}

De nye retningslinjene for evaluering av bacheloroppgaver ved Høgskolen i Østfold/IT legger større vekt på hoveddokumentet enn før. Denne rapporten er resultatet av et prosjekt der formålet var å gi studentene en mulighet for å forenkle og forbedre dokumentproduksjonen. Rapporten er en selvforklarende mal som tar for seg innhold, struktur og layout av hoveddookumentet i bacheloroppgaven. I tillegg er den et konkret eksempel på hvordan man kan bruke \LaTeX~ som dokumentverktøy. Dokumentet er en mal, dvs. et stilsett som brukes for å gi dokumentet ønsket layout.  Det blir gitt eksempler på de viktigste teknikkene, slik som bruk av kryssreferanser, kildereferanser, figurer og tabeller, og eksempler på formattering av spesielle elementer, som lister, sitater, definisjoner og matematiske uttrykk. I de tilfellene eksemplene ikke er selvforklarende, blir det gitt råd om hvordan man skal få det til. Intensjonen er at malen kan brukes for alle de tre hovertypene av bachelorprosjekter ved HiØ/IT: Utredninger, mediaproduksjoner, og utvikling av programvare, maskinvare eller systemer. Der det er naturlig å differensiere innholdet i de enkelte kapitlene, blir det skissert mulige løsninger for alle typene prosjekt. Formgivingen er enkel, oversiktlig og tradisjonell. Utgangspunktet for strukturen er den generiske oppbyggingen av et teknisk-vitenskapelig dokument, slik det er beskrevet i {\em Mayfield Handbook of Technical \& Scientific Writing}.
Innholdet i denne rapporten er en (kanskje forvirrende) blanding av generiske retningslinjer og konkret eksemplifisering relatert til prosjektet med å utvikle malen.