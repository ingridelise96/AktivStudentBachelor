\documentclass[11pt,a4paper]{report} 
\usepackage[utf8]{inputenc} 
\usepackage[norsk]{babel} 

\begin{document}

\pagestyle{empty}

\title{
{\Huge  \vspace{-2cm} Prosjektkontrakt}\\
\vspace{2cm}
Bacheloroppgaven\\
Avdeling for Informasjonsteknologi\\
Høgskolen i Østfold
}

\author{
Gruppe BO??-??: \\
Navn 1\\
Navn 2\\
...\\
Navn 3 \bigskip\\
Oppdragsgiver: \\
Bedrift\\
Kontakperson(er)\bigskip\\
Veileder: \\
Navn\bigskip}

\maketitle

\subsection*{Prosjektbeskrivelse}

\texttt{\textit{Fyll inn: Kort  beskrivelse av prosjektet, men noe mer fyldig i det originale prosjektforslaget. Hva er formålet med prosjektet, for oppdragsgiver? Hva skal leveres? Hva er kravene/rammene? Hvordan skal prosjektet gjennomføres?}}

\subsection*{Studentgruppen}

Studentgruppen har selv det fulle ansvar når det gjelder organisering og gjennomføring av prosjektet. Ikke minst innebærer dette å følge opp tidsfrister. Kontakt med oppdragsgiver og veileder er også studentenes ansvar. De må også holde seg oppdatert på beskjeder og meldinger fra fagansvarlig (som regel pr. epost). Studentene skal selv fordele arbeidet i gruppa, og eventuelt velge prosjektleder (kan byttes på underveis hvis ønskelig). 

Gruppen har rett til veiledning fra en eller flere av HiØ/IT sine fagansatte tilsvarende ca. en halv time pr. uke gjennom hele semesteret. Studentene skal også ha fri tilgang til program- og maskinvare og annet nødvendig utstyr. Er det behov for reiser og liknende, skal dette dekkes etter avtale med oppdragsgiver og/eller HiØ/IT. 

\subsection*{Oppdragsgiver}

Det er viktig at oppdragsgiver er innforstått med rammene rundt bacheloroppgaven. Oppdragsgiver må ikke definere en oppgave som ellers ville bli utført av eksterne konsulenter o.l. Det vil alltid være en risiko for at studentene ikke kommer helt i mål, eller leverer noe som ikke nødvendigvis er ferdige, operative løsninger.

Oppdragsgiver skal gi rammene for prosjektet, dvs. hjelpe til med å definere formål, leveranser, og evt. metode. Det er svært viktig å gi studentene nok informasjon, og evt. tilgang til utstyr, programvare o.l.

Oppdragsgiver må være forberedt på å svare på spørsmål på kort varsel, og sørge for å være oppdatert på prosjektets framdrift. Det kan også være aktuelt å hjelpe til med å revidere og endre prosjektplanen underveis.

Det er vanlig å ha regelmessige møter med gruppa, med en frekvens som passer prosjektets egenart. 

\subsection*{Veileder}

Veileder skal hjelpe til med: 1) Gjennomføring og dokumentasjon, og 2) Faglige spørsmål (i de tilfellene der det faglige blir for smalt og spesielt, må studentene enten skaffe faglig hjelp fra oppdragsgiver, eller fra annet hold). 

\subsection*{Rettigheter}

Deltagere med rettigheter i prosjektarbeidet er ideforfatter, utøvende medarbeidere og oppdragsgiver.
Ideforfatter kan være veileder, en eller flere studenter, oppdragsgiver eller en kombinasjon av disse. 
Utøvende medarbeidere er studentene og i noen grad veileder. 
Oppdragsgiver kan enten være en ekstern institusjon/bedrift eller en avdeling ved Høgskolen i Østfold. 

Følgende rettigheter er knyttet til prosjektdeltagelse:


\begin{description}
	\item[Opphavsrett] 
    Rett til benevnelse ved publikasjon og eiendomsrett til originalt åndsverk (som sikres ved publikasjon). Opphavsrett innehas av ideforfatter og utøvende medarbeidere. 
	\item[Disposisjonsrett]
    Rett til videreutvikling av et avsluttet arbeid. Denne innehas av oppdragsgiver. 
	\item[Eiendomsrett]
    Rett til oppbevaring og omsetning av et avsluttet arbeid og innehas av oppdragsgiver. 
\end{description}

Avvik fra disse prinsippene kan avtales før godkjenning av prosjekter etter ønsker fra oppdragsgiver. Det skal da lages en avtale som inngås av alle deltagerne i prosjektet. Tvister om rettigheter avgjøres etter deltagernes samtykke av avdelingsstyret, subsidiært ved forfølgelse i det alminnelige rettsvesen. 

Hvis oppdragsgiver ikke ønsker at resultatene eller deler av disse skal offentliggjøres, må dette avklares ved oppstart av prosjektet.


\vspace{1cm}
\begin{center}
Underskrifter
\end{center}


\end{document}
