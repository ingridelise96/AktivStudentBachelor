\documentclass[12pt,a4paper]{report} 
\usepackage[utf8]{inputenc} 
\usepackage[norsk]{babel} 

\begin{document}

\pagestyle{empty}

\title{
{\Huge  \vspace{-2cm} Gruppekontrakt}\\
\vspace{2cm}
Bacheloroppgaven\\
Avdeling for Informasjonsteknologi\\
Høgskolen i Østfold
}

\author{
Gruppe BO??-?? \\
Navn 1\\
Navn 2\\
..\\
Navn N}
\maketitle

\begin{description}
  \item[\bfseries § 1 Demokrati] 
  Alle avgjørelser skal baseres på flertallet i gruppa. Medlemmene bør likevel prøve å komme fram til fullstendig enighet.
  \item[\bfseries § 2 Plikter] Det er møte- og forberedelsesplikt for alle. Ved sykdom eller annen gyldig fraværsgrunn skal ett eller flere av de andre medlemmene varsles så tidlig som mulig. Medlemmene forplikter seg også til å levere avtalt arbeid til rett tid, eller si fra i god tid hvis forsinkelser oppstår.
  \item[\bfseries § 3 Møtetider] Møtetider bør avtales slik at ingen i gruppen hindres i foreberedelser eller oppmøte.
  \item[\bfseries § 4 Utgifter] Alle utgifter i prosjektet (reising, kopiering, inngangspenger, tidsskrifter etc.) skal deles likt mellom medlemmene. Utgiftene skal kunne dokumenteres.
  \item[\bfseries § 5 Tvister] Uenigheter og problemer bør bli behandlet og løst i plenum, internt i gruppen. Hvis dette slår feil, skal veileder kontaktes.
  \item[\bfseries § 6 Arbeidsdeling] Alt arbeid skal deles så likt og rettferdig som mulig. Alle har rett til å påpeke forhold de mener er urettferdige. Totalt må hvert medlem bruke  ca.\ 500 timer på prosjektet. Dette inkludert alt relatert arbeid, slik som f.eks. møter med oppdragsgiver og veileder. Det skal føres individuelle timelister som dokumenterer tidsbruken.
  \item[\bfseries § 7 Samarbeid] I en gruppe gjelder en for alle, alle for en. Det er viktig at alle forsøker å dele tid, arbeid, erfaringer og kunnskap. Det er en selvfølge å hjelpe til hvis noen står fast.
   \item[\bfseries § 8 Skjevfordeling] Ved en åpenbar skjevfordeling kan det være aktuelt å gi ulike karakterer i gruppen. Dette vil i så fall være etter gruppens eget ønske. Gruppen må fange opp slike problemer så raskt som mulig, og prøve å løse de umiddelbart, gjerne i samråd med veileder og evt. fagansvarlig.
    \item[\bfseries § 9 Kontraktsbrudd] Alvorlige og/eller gjentatte regelbrudd kan føre til eksklusjon fra gruppen. Dette må i så fall skje i samråd med veileder og fagansvarlig.
\end{description}

\vspace{1cm}
\begin{center}
Underskrifter:
\end{center}


\end{document}
